\documentclass[dvipdfmx,12pt,beamer]{standalone}
\ifstandalone
	%%%% 和文用(よくわかってないです) %%%%
\usepackage{bxdpx-beamer} %ナビゲーションシンボル(?)を機能させる
\usepackage{pxjahyper} %しおりの日本語対応
\usepackage{minijs} %和文メトリックの調整
\renewcommand{\kanjifamilydefault}{\gtdefault} %和文規定をゴシックに(変えたければrenewcommand で調整)
%%%%%%%%%%

%%%% TikZ %%%%
\usepackage{tikz}
\usetikzlibrary{calc,decorations.pathreplacing,quotes,positioning,shapes,fit,arrows,backgrounds,tikzmark}
%%%%%%%%

%%%% ファイル分割用のパッケージ %%%%
\usepackage{standalone}
\usepackage{import}
%%%%%%%%

%%%% スライドの見た目 %%%%%
\usetheme{Metropolis}
\usepackage{xcolor}
%\usefonttheme{professionalfonts}
%%%%%%%%%%

%%%%% フォント基本設定 %%%%%
\usepackage[T1]{fontenc}%8bit フォント
\usepackage{textcomp}%欧文フォントの追加
\usepackage[utf8]{inputenc}%文字コードをUTF-8
\usepackage{otf}%otfパッケージ
%\usepackage{lxfonts}%数式・英文ローマン体を Lxfont にする
\usepackage{bm}%数式太字
%%%%%%%%%%

%%%% Standaloneかどうかでimportのパスを切り替える %%%%
\providecommand{\ImportStandalone}[3]{
	\IfStandalone{
		\import{#2}{#3}
		}{
		\import{#1#2}{#3}
	}
}
%%%%%%%%

\fi

\begin{document}
\begin{frame}{TNFAの構築}
  展開されたパターンからTNFAを構築
  \begin{itemize}
    \item \alert<2>{演算子$|$を親として,構文木(Parse Tree)を構築}
      \begin{itemize}
        \item この構文木での各頂点と根の間にある$|$の数を\\対応するパターンの\alert{深さ}(depth)とする
      \end{itemize}
    \item \alert<3>{構文木を利用し,前述の規則に従ってTNFAを構築}
  \end{itemize}
  {\small 例 $A(AB|B?)(B?.*|AB)C$ }\\
  \onslide*<1>{\vspace{4cm}}
  \onslide*<2>{
    \centering
  \scalebox{.7}{
\begin{tikzpicture}[state/.style={circle, draw, minimum size=.8cm}, ]
  \node(ptree) [state] {$\cdot$} [level distance=1cm, level 1/.style={sibling distance=4cm}, level 2/.style={sibling distance=2cm}, level 3/.style={sibling distance=1cm}]
    child {node [state] {$A$}}
    child {node [state] {$|$}
      child {node [state] {$\cdot$} 
        child {node [state] {$A$}} 
        child {node [state] {$B$}}
      }
      child[level distance=2cm] {node [state] {$B?$} }
    }
    child {node [state] {$|$}
      child {node [state] {$\cdot$}
        child {node [state] {$B?$}}
        child {node [state] {$.^*$}}
      }
      child {node [state] {$\cdot$}
        child {node [state] {$A$}}
        child {node [state] {$B$}}
      }
    }
  child {node [state] {$C$}};
  \node [above =.5cm of ptree-1] {depth $0$};
  \node [below =.1cm of ptree-1] {depth $1$};
  \draw ($(ptree-1)+(-1,-.5)$) -- ($(ptree-4)+(1,-.5)$);
  \draw ($(ptree-1)+(-1,1.5)$) -- ($(ptree-4)+(1,1.5)$);
  \draw ($(ptree-1)+(-1,-2.5)$) -- ($(ptree-4)+(1,-2.5)$);
\end{tikzpicture}
}
}
\onslide*<3>{
  \centering
  \scalebox{.7}{
    \ImportStandalone{chapter/algorithm/}{../../figure/}{state}
  }
}
\end{frame}
\end{document}
