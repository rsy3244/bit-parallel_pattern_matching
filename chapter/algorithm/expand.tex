\documentclass[dvipdfmx,12pt,beamer]{standalone}
\ifstandalone
	%%%% 和文用(よくわかってないです) %%%%
\usepackage{bxdpx-beamer} %ナビゲーションシンボル(?)を機能させる
\usepackage{pxjahyper} %しおりの日本語対応
\usepackage{minijs} %和文メトリックの調整
\renewcommand{\kanjifamilydefault}{\gtdefault} %和文規定をゴシックに(変えたければrenewcommand で調整)
%%%%%%%%%%

%%%% TikZ %%%%
\usepackage{tikz}
\usetikzlibrary{calc,decorations.pathreplacing,quotes,positioning,shapes,fit,arrows,backgrounds,tikzmark}
%%%%%%%%

%%%% ファイル分割用のパッケージ %%%%
\usepackage{standalone}
\usepackage{import}
%%%%%%%%

%%%% スライドの見た目 %%%%%
\usetheme{Metropolis}
\usepackage{xcolor}
%\usefonttheme{professionalfonts}
%%%%%%%%%%

%%%%% フォント基本設定 %%%%%
\usepackage[T1]{fontenc}%8bit フォント
\usepackage{textcomp}%欧文フォントの追加
\usepackage[utf8]{inputenc}%文字コードをUTF-8
\usepackage{otf}%otfパッケージ
%\usepackage{lxfonts}%数式・英文ローマン体を Lxfont にする
\usepackage{bm}%数式太字
%%%%%%%%%%

%%%% Standaloneかどうかでimportのパスを切り替える %%%%
\providecommand{\ImportStandalone}[3]{
	\IfStandalone{
		\import{#2}{#3}
		}{
		\import{#1#2}{#3}
	}
}
%%%%%%%%

\fi

\begin{document}
\begin{frame}{パターンの展開}
  拡張パターンの一部の記号を等価な別の表現に展開\\
  拡張パターンの記号(再掲)
  \begin{center}
    \small
    \begin{tabular}{|c|l|}\hline
      \alert{$.$} & 任意の文字 \\  & 例 $.B = \{AB, BB, CB, \ldots \}$ \\\hline
      \alert{$?$\tikzmark{optional}} & 0,1回の文字 \\  & 例 $A?B = \{AB, B \}$ \\\hline
      \alert{${}^*$} & 0回以上の文字の繰り返し \\  & 例 $A^* = \{\epsilon, A, AA, \ldots \}$ \\\hline
      \alert{${}^+$\tikzmark{rep1}} & 1回以上の文字の繰り返し \\  & 例 $A^+ = \{A, AA, \ldots \}$ \\\hline
      \alert{$\{x,y\}$\tikzmark{bndrep}} & $x$回以上,$y$回以下の文字の繰り返し \\  & 例 $A\{2,4\} = \{AA, AAA, AAAA \}$ \\\hline
    \end{tabular}
  \end{center}

  \begin{tikzpicture}[overlay, remember picture]
    \node<2> [ellipse callout, draw, fill=blue!5, callout absolute pointer={(pic cs:optional)}] at ($(pic cs:optional) + (1,-2)$) {$A? = (\epsilon | A)$に変換};
    \node<3> [ellipse callout, draw, fill=blue!5, callout absolute pointer={($(pic cs:rep1) + (0,.2)$)}] at ($(pic cs:rep1) + (3,2)$) {$A^+ = (AA^*)$に変換};
    \node<4> [ellipse callout, draw, fill=blue!5, callout absolute pointer={($(pic cs:bndrep)+(0,.2)$)}] at ($(pic cs:bndrep) + (3,2)$) {$A\{x,y\} = ((A?)^{y-x}A^x)$に変換};
    %\node<3> [ellipce callout, 
  \end{tikzpicture}


  
\end{frame}
\end{document}
