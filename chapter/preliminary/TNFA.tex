\documentclass[dvipdfmx,12pt,beamer]{standalone}
\ifstandalone
	%%%% 和文用(よくわかってないです) %%%%
\usepackage{bxdpx-beamer} %ナビゲーションシンボル(?)を機能させる
\usepackage{pxjahyper} %しおりの日本語対応
\usepackage{minijs} %和文メトリックの調整
\renewcommand{\kanjifamilydefault}{\gtdefault} %和文規定をゴシックに(変えたければrenewcommand で調整)
%%%%%%%%%%

%%%% TikZ %%%%
\usepackage{tikz}
\usetikzlibrary{calc,decorations.pathreplacing,quotes,positioning,shapes,fit,arrows,backgrounds,tikzmark}
%%%%%%%%

%%%% ファイル分割用のパッケージ %%%%
\usepackage{standalone}
\usepackage{import}
%%%%%%%%

%%%% スライドの見た目 %%%%%
\usetheme{Metropolis}
\usepackage{xcolor}
%\usefonttheme{professionalfonts}
%%%%%%%%%%

%%%%% フォント基本設定 %%%%%
\usepackage[T1]{fontenc}%8bit フォント
\usepackage{textcomp}%欧文フォントの追加
\usepackage[utf8]{inputenc}%文字コードをUTF-8
\usepackage{otf}%otfパッケージ
%\usepackage{lxfonts}%数式・英文ローマン体を Lxfont にする
\usepackage{bm}%数式太字
%%%%%%%%%%

%%%% Standaloneかどうかでimportのパスを切り替える %%%%
\providecommand{\ImportStandalone}[3]{
	\IfStandalone{
		\import{#2}{#3}
		}{
		\import{#1#2}{#3}
	}
}
%%%%%%%%

\fi

\begin{document}
\begin{frame}{非決定性有限オートマトン({\small Nondeterministic Finite Automaton})}
  \begin{block}{\alert{非決定性有限オートマトン}(NFA)}
    以下の要素を持つ有限オートマトンの一種\\
    \begin{enumerate}
      \item 状態集合$Q$
      \item 入力アルファベット$\Sigma$
      \item 遷移関数$\delta : \left(\Sigma \cup \epsilon \right) \rightarrow 2^Q$
      \item 開始状態$s_0 \in Q$
      \item 受理状態集合$T \subset Q$\\
    \end{enumerate}
  \end{block}
  \begin{itemize}
    \item 文字列を受け取ることで,\\受理できるかを判定することができる.
    \item 決定性有限オートマトン({\small Deterministic Finite Automaton})とは異なり,空文字での遷移を持つことができる.
  \end{itemize}
    

\end{frame}
\begin{frame}{トンプソンオートマトン(Thompson NFA)(1/2)}
  \alert{トンプソンオートマトン}(TNFA): 以下の規則によって\\生成された,非決定性有限オートマトン(NFA)
  \begin{table}
    \centering
    \begin{tabular}{cccc}
      $N(\epsilon)$ & $N(\alpha)$ & $N(\alpha?)$ & $N(\alpha*)$ \\
      \scalebox{.7}{
        \ImportStandalone{chapter/preliminary/}{TNFA/}{epsilon}
    }
      &
      \scalebox{.7}{
        \ImportStandalone{chapter/preliminary/}{TNFA/}{atom}
    }
      &
      \scalebox{.7}{
        \ImportStandalone{chapter/preliminary/}{TNFA/}{optional}
      }
      &
      \scalebox{.7}{
        \ImportStandalone{chapter/preliminary/}{TNFA/}{rep}
      }
    \end{tabular}
  \end{table}
\end{frame}
\begin{frame}{トンプソンオートマトン(Thompson NFA)(2/3)}
  \alert{トンプソンオートマトン}(TNFA): 以下の規則によって\\生成された,非決定性有限オートマトン(NFA)
  \begin{table}
    \centering
    \begin{tabular}{cc}
      $N((R_1\ldots R_n))$ & $N((R_1 \mid \ldots \mid R_n))$\\
      \scalebox{.5}{
        \ImportStandalone{chapter/preliminary/}{TNFA/}{concat}
    }
      &
      \scalebox{.5}{
        \ImportStandalone{chapter/preliminary/}{TNFA/}{union}
      }
    \end{tabular}
  \end{table}
  \begin{table}
    \centering
    \begin{tabular}{c}
      $N((R)^*)$ \\
    \scalebox{.7}{
      \ImportStandalone{chapter/preliminary/}{TNFA/}{kleene}
    }
    \end{tabular}
  \end{table}
\end{frame}
\begin{frame}{トンプソンオートマトン(Thompson NFA)(3/4)}
  遷移となる辺を以下のように定義
  \begin{description}
    \item[$\alpha\mathchar`-edge$ : ]  文字を受け取ったときに遷移する辺
    \item[$\epsilon\mathchar`-edge$ : ]  空文字で遷移する辺
  \end{description}
    \begin{description}
      \item[$s\mathchar`-edge$ : ] $1$状態から深さが等しい状態へ分岐する$\epsilon\mathchar`-edge$
      \item[$g\mathchar`-edge$ : ] $1$状態に深さが等しい状態へ集約する$\epsilon\mathchar`-edge$
      \item[$p\mathchar `-edge$ : ] 状態の添字が増加するような$\epsilon\mathchar`-edge$
      \item[$b\mathchar`-edge$ : ] $\epsilon\mathchar`-edge$のうち,$p\mathchar`-edge$でないもの
    \end{description}
\end{frame}
\begin{frame}{トンプソンオートマトン(Thompson NFA)(4/4)}
  \newcommand{\TNFAActivateEdgeLabel}{\relax}
  \begin{table}
    \centering
    \begin{tabular}{cccc}
      \scalebox{.7}{
        \ImportStandalone{chapter/preliminary/}{TNFA/}{epsilon}
    }
      &
      \scalebox{.7}{
        \ImportStandalone{chapter/preliminary/}{TNFA/}{atom}
    }
      &
      \scalebox{.7}{
        \ImportStandalone{chapter/preliminary/}{TNFA/}{optional}
      }
      &
      \scalebox{.7}{
        \ImportStandalone{chapter/preliminary/}{TNFA/}{rep}
      }
    \end{tabular}
  \end{table}
  \begin{table}
    \centering
    \begin{tabular}{cc}
      \scalebox{.5}{
        \ImportStandalone{chapter/preliminary/}{TNFA/}{union}
      }
      &
      \scalebox{.7}{
        \ImportStandalone{chapter/preliminary/}{TNFA/}{kleene}
      }
    \end{tabular}
  \end{table}
\end{frame}



\end{document}
