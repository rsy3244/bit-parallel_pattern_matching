\documentclass[dvipdfmx,beamer]{standalone}
\ifstandalone
	%%%% 和文用(よくわかってないです) %%%%
\usepackage{bxdpx-beamer} %ナビゲーションシンボル(?)を機能させる
\usepackage{pxjahyper} %しおりの日本語対応
\usepackage{minijs} %和文メトリックの調整
\renewcommand{\kanjifamilydefault}{\gtdefault} %和文規定をゴシックに(変えたければrenewcommand で調整)
%%%%%%%%%%

%%%% TikZ %%%%
\usepackage{tikz}
\usetikzlibrary{calc,decorations.pathreplacing,quotes,positioning,shapes,fit,arrows,backgrounds,tikzmark}
%%%%%%%%

%%%% ファイル分割用のパッケージ %%%%
\usepackage{standalone}
\usepackage{import}
%%%%%%%%

%%%% スライドの見た目 %%%%%
\usetheme{Metropolis}
\usepackage{xcolor}
%\usefonttheme{professionalfonts}
%%%%%%%%%%

%%%%% フォント基本設定 %%%%%
\usepackage[T1]{fontenc}%8bit フォント
\usepackage{textcomp}%欧文フォントの追加
\usepackage[utf8]{inputenc}%文字コードをUTF-8
\usepackage{otf}%otfパッケージ
%\usepackage{lxfonts}%数式・英文ローマン体を Lxfont にする
\usepackage{bm}%数式太字
%%%%%%%%%%

%%%% Standaloneかどうかでimportのパスを切り替える %%%%
\providecommand{\ImportStandalone}[3]{
	\IfStandalone{
		\import{#2}{#3}
		}{
		\import{#1#2}{#3}
	}
}
%%%%%%%%

\fi

\begin{document}
\begin{frame}{ネットワーク表現}
	ネットワーク表現とは,正規表現からクリーネ閉包を除いた表現.
	\begin{itemize}
		\item $\alpha$は集合$\{\alpha\}$を表す
			\begin{itemize}
				\item $a$は,$\{a\}$の正規表現
			\end{itemize}
		\item $R_1R_2$は連接を表す
			\begin{itemize}
				\item $(ca\mid ab)c$は,$\{cac, abc\}$の正規表現
			\end{itemize}
    \item $R_1\alert{|}R_2$は$R_1$と$R_2$の和集合を表す
			\begin{itemize}
				\item $abc\mid ba$は,$\{abc, ba\}$の正規表現
			\end{itemize}
    \item $R^{\alert{*}}$は$R$の$0$回以上の繰り返しを表す
			\begin{itemize}
				\item $a^*$は,$\{\epsilon, a, aa, \cdots\}$の正規表現
			\end{itemize}
	\end{itemize}
	
\end{frame}
\end{document}
