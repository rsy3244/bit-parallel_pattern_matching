\documentclass[dvipdfmx,12pt,beamer]{standalone}
\ifstandalone
	\input{../../preamble.tex}
\fi

\begin{document}
\begin{frame}{クラス}
  \alert{クラス(Class)}
  \begin{itemize}
    \item ある表現が扱えるような文字列の全体集合
    \item 正規表現クラスは$\textsf{REG}$
  \end{itemize}
  \alert{ネットワーク表現(Network Expressions)}
  \begin{itemize}
    \item 正規表現からクリーネ閉包を除いた表現
    \item クラスは$\textsf{NET}$
    \item $\textsf{NET} \subset \textsf{REG} \subset \Sigma^*$
  \end{itemize}
\end{frame}

\begin{frame}{拡張パターン}
  \alert{拡張パターン(Extended String Pattern)} : \\ \, 文字列に以下の記号を加えて表現される表現
  \begin{center}
    \small
    \begin{tabular}{|c|l|}\hline
      \alert{$.$} & 任意の文字 \\  & 例 $.b = \{ab, bb, cb, \ldots \}$ \\\hline
      \alert{$?$} & 0,1回の文字 \\  & 例 $a?b = \{ab, b \}$ \\\hline
      \alert{${}^*$} & 0回以上の文字の繰り返し \\  & 例 $a^* = \{\epsilon, a, aa, \ldots \}$ \\\hline
      \alert{${}^+$} & 1回以上の文字の繰り返し \\  & 例 $a^+ = \{a, aa, \ldots \}$ \\\hline
      \alert{$\{x,y\}$} & $x$回以上,$y$回以下の文字の繰り返し \\  & 例 $a\{2,4\} = \{aa, aaa, aaaa \}$ \\\hline
    \end{tabular}
  \end{center}
\end{frame}

\begin{frame}{$(3/3)$}

  \begin{itemize}
    \item 拡張表現のクラスは$\textsf{EXT}$
  \end{itemize}
  \begin{center}
    \begin{tikzpicture}
      \node(NET) [circle, draw, minimum size=1cm] at (0,0) {$\textsf{NET}$};
      \node(EXNET) [left=of NET] {$\textsf{EXNET}$};
      \node(extnet) [rounded rectangle, draw, fit= (NET) (EXNET)] {};
      \node(REG) at (-3.5, 2) {$\textsf{REG} = \textsf{EXT}$};
      \begin{scope}[on background layer, remember picture]
        \node(reg) [rounded rectangle, fill={rgb:red,1;blue,10;white,100}, draw, fit=(NET) (EXNET) (extnet) (REG)] {};
      \end{scope}
    \end{tikzpicture}
  \end{center}

\end{frame}
\end{document}
