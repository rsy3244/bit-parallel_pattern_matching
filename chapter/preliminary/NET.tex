\documentclass[dvipdfmx,beamer]{standalone}
\ifstandalone
	\input{preamble.tex}
\fi

\begin{document}
\begin{frame}{ネットワーク表現}
	ネットワーク表現とは,正規表現からクリーネ閉包を除いた表現.
	\begin{itemize}
		\item $\alpha$は集合$\{\alpha\}$を表す
			\begin{itemize}
				\item $a$は,$\{a\}$の正規表現
			\end{itemize}
		\item $R_1R_2$は連接を表す
			\begin{itemize}
				\item $(ca\mid ab)c$は,$\{cac, abc\}$の正規表現
			\end{itemize}
    \item $R_1\alert{|}R_2$は$R_1$と$R_2$の和集合を表す
			\begin{itemize}
				\item $abc\mid ba$は,$\{abc, ba\}$の正規表現
			\end{itemize}
    \item $R^{\alert{*}}$は$R$の$0$回以上の繰り返しを表す
			\begin{itemize}
				\item $a^*$は,$\{\epsilon, a, aa, \cdots\}$の正規表現
			\end{itemize}
	\end{itemize}
	
\end{frame}
\end{document}
