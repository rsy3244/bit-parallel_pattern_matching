\documentclass[dvipdfmx,beamer]{standalone}
\ifstandalone
	%%%% 和文用(よくわかってないです) %%%%
\usepackage{bxdpx-beamer} %ナビゲーションシンボル(?)を機能させる
\usepackage{pxjahyper} %しおりの日本語対応
\usepackage{minijs} %和文メトリックの調整
\renewcommand{\kanjifamilydefault}{\gtdefault} %和文規定をゴシックに(変えたければrenewcommand で調整)
%%%%%%%%%%

%%%% TikZ %%%%
\usepackage{tikz}
\usetikzlibrary{calc,decorations.pathreplacing,quotes,positioning,shapes,fit,arrows,backgrounds,tikzmark}
%%%%%%%%

%%%% ファイル分割用のパッケージ %%%%
\usepackage{standalone}
\usepackage{import}
%%%%%%%%

%%%% スライドの見た目 %%%%%
\usetheme{Metropolis}
\usepackage{xcolor}
%\usefonttheme{professionalfonts}
%%%%%%%%%%

%%%%% フォント基本設定 %%%%%
\usepackage[T1]{fontenc}%8bit フォント
\usepackage{textcomp}%欧文フォントの追加
\usepackage[utf8]{inputenc}%文字コードをUTF-8
\usepackage{otf}%otfパッケージ
%\usepackage{lxfonts}%数式・英文ローマン体を Lxfont にする
\usepackage{bm}%数式太字
%%%%%%%%%%

%%%% Standaloneかどうかでimportのパスを切り替える %%%%
\providecommand{\ImportStandalone}[3]{
	\IfStandalone{
		\import{#2}{#3}
		}{
		\import{#1#2}{#3}
	}
}
%%%%%%%%

\fi

\begin{document}
\begin{frame}{クラス}
  \alert{クラス}
  \begin{itemize}
    \item ある表現が扱えるような文字列の全体集合
    \item 正規表現クラスは$\textsf{REG}$
  \end{itemize}
  \alert{ネットワーク表現}
  \begin{itemize}
    \item 正規表現からクリーネ閉包を除いた表現
    \item クラスは$\textsf{NET}$
    \item $\textsf{NET} \subset \textsf{REG} \subset \Sigma^*$
  \end{itemize}
\end{frame}

\begin{frame}{拡張表現}
  \alert{拡張表現} : 正規表現に以下の記号を加える.
    %\begin{description}
    %  \item[$.$] 任意の文字
    %    \begin{itemize} \item $.b$ は $\{ab, bb, cb,\ldots\}$を表す. \end{itemize}
    %  \item[$?$] 0,1回の文字
    %    \begin{itemize} \item $a?b$ は $\{ab, b\}$を表す. \end{itemize}
    %    \item[${}^*, {}^+$] 0回以上,1回以上の文字の繰り返し
    %    \begin{itemize} \item $a^*$, $a^+$ は それぞれ$\{\epsilon, a, aa, \ldots\}, \{a, aa, \ldots\}$を表す. \end{itemize}
    %  \item[$\{x,y\}$] $x$回以上,$y$回以下の文字の繰り返し
    %  \begin{itemize} \item $a\{2,4\}$ は $\{aa, aaa, aaaa\}$を表す. \end{itemize}
    %\end{description}
    \begin{tabular}{|c|l|}\hline
      \alert{$.$} & 任意の文字 \\ 例 & $.b = \{ab, bb, cb, \ldots \}$ \\\hline
      \alert{$?$} & 0,1回の文字 \\ 例 & $a?b = \{ab, b \}$ \\\hline
      \alert{${}^*$} & 0回以上の文字の繰り返し \\ 例 & $a^* = \{\epsilon, a, aa, \ldots \}$ \\\hline
      \alert{${}^+$} & 1回以上の文字の繰り返し \\ 例 & $a^+ = \{a, aa, \ldots \}$ \\\hline
      \alert{$\{x,y\}$} & $x$回以上,$y$回以下の文字の繰り返し \\ 例 & $a\{2,4\} = \{aa, aaa, aaaa \}$ \\\hline
    \end{tabular}
\end{frame}

\begin{frame}{$(3/3)$}
  \begin{itemize}
    \item 前述の拡張表現のクラスは$\textsf{EXT}$

  \end{itemize}
  \begin{center}
    \begin{tikzpicture}
      \node(NET) [circle, draw, minimum size=1cm] at (0,0) {$\textsf{NET}$};
      \node(EXNET) [left=of NET] {$\textsf{EXNET}$};
      \node(extnet) [rounded rectangle, draw, fit= (NET) (EXNET)] {};
      \node(REG) at (-3.5, 2) {$\textsf{REG} = \textsf{EXT}$};
      \begin{scope}[on background layer, remember picture]
        \node(reg) [rounded rectangle, fill={rgb:red,1;blue,10;white,100}, draw, fit=(NET) (EXNET) (extnet) (REG)] {};
      \end{scope}
    \end{tikzpicture}
  \end{center}

\end{frame}
\end{document}
