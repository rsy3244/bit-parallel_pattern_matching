\documentclass[dvipdfmx,12pt,notheorems]{beamer} %dvipdfmxを使う場合は指定

%%%% 和文用(よくわかってないです) %%%%
\usepackage{bxdpx-beamer} %ナビゲーションシンボル(?)を機能させる
\usepackage{pxjahyper} %しおりの日本語対応
\usepackage{minijs} %和文メトリックの調整
\renewcommand{\kanjifamilydefault}{\gtdefault} %和文規定をゴシックに(変えたければrenewcommand で調整)
%%%%%%%%%%

%%%% TikZ %%%%
\usepackage{tikz}
\usetikzlibrary{calc,decorations.pathreplacing,quotes,positioning,shapes,fit,arrows,backgrounds,tikzmark}
%%%%%%%%

%%%% ファイル分割用のパッケージ %%%%
\usepackage{standalone}
\usepackage{import}
%%%%%%%%

%%%% スライドの見た目 %%%%%
\usetheme{Metropolis}
\usepackage{xcolor}
%\usefonttheme{professionalfonts}
%%%%%%%%%%

%%%%% フォント基本設定 %%%%%
\usepackage[T1]{fontenc}%8bit フォント
\usepackage{textcomp}%欧文フォントの追加
\usepackage[utf8]{inputenc}%文字コードをUTF-8
\usepackage{otf}%otfパッケージ
%\usepackage{lxfonts}%数式・英文ローマン体を Lxfont にする
\usepackage{bm}%数式太字
%%%%%%%%%%

%%%% Standaloneかどうかでimportのパスを切り替える %%%%
\providecommand{\ImportStandalone}[3]{
	\IfStandalone{
		\import{#2}{#3}
		}{
		\import{#1#2}{#3}
	}
}
%%%%%%%%

\usepackage{biblatex}
\addbibresource{ref.bib}
\let\oldfootnotesize\footnotesize
\renewcommand*{\footnotesize}{\oldfootnotesize\tiny}

\title[略タイトル]{Bit-parallelを用いたパターンマッチアルゴリズムの検証} %タイトルの行間がおかしい
\author[Mitsuyoshi]{光吉 健汰\inst{1}}
\institute[IKN]{\inst{1}北海道大学工学部 情報エレクトロニクス学科 情報理工学コース 4年\\
情報知識ネットワーク研究室}
\date{\today}%日付

\begin{document}
\begin{frame}{{\small Bit-parallelを用いた正規表現のマッチングアルゴリズムの検証}\\
		{\scriptsize 光吉健汰(情報理工学コース4年,情報知識ネットワーク研究室)}
	}
	\begin{description}[背景:]
		\item[背景:] 正規表現の高速なマッチングアルゴリズム
		\item[目的:] Kaneta+[2010]\footfullcite{Kaneta:2010:FBM:1928328.1928377}が提案したマッチングアルゴリズムをCPU上で実装し,パフォーマンスを検証
	\end{description}
	\begin{itemize}
		\item Kaneta+[2010]が,ハードウェア指向のマッチングアルゴリズムを提案
			\begin{itemize}
				\item FPGA上で実装を行なっていた
			\end{itemize}
		\item 現代のCPUは十分な幅($64\sim256$bit)のレジスタを持つ
			\begin{itemize}
				\item CPU上で実装し,パフォーマンスを検証
			\end{itemize}
	\end{itemize} 
	
\end{frame}

\begin{frame}{正規表現のマッチングアルゴリズム}
	\begin{itemize}
		\item 正規表現で表されるパターンからThompson NFAを構築
		\item Thompson NFAの模倣を複数のビットマスクとビット演算によって実現
			\begin{itemize}
				\item 分岐する遷移 {\scriptsize $D \gets D \mid \left( \left( BLK_s[k] - \left( D \& SRC_s[k] \right) \right) \& DST_s[k] \right)$}\\
				\item 集約する遷移 {\scriptsize $D \gets D \mid \left( \left( BLK_g[k] + \left( D \& SRC_g[k] \right) \right) \& DST_g[k] \right)$}\\
				\item 伝搬する遷移 {\scriptsize $A \gets \left( D \& BLK_p[k] \right) \mid DST_p[k]$}\\
					\hspace{2.2cm}{\scriptsize $D \gets D \mid \left( BLK_p[k] \& \left( \left( \sim \left( A - SRC_p[k] \right) \right) \oplus A \right) \right)$}\\
			\end{itemize}
	\end{itemize}
  \begin{tikzpicture}
    \node(pat) at (0,0) {$A(AB|B?)(B?.^*|AB)C$};
    \node(expand) [right =of pat] {$A(AB|(\epsilon |B))((\epsilon |B).^*|AB)C$};
    \node(tnfa) [below =of pat] {\scalebox{.25}{\import{../figure/}{state}}};
    \node(bitmask) [below =of expand] {\scalebox{.45}{\import{../figure/}{bitmask}}};

    \draw[->, very thick] (pat) -- (expand);
    \draw[->, very thick] (expand.south west) -- (tnfa);
    \draw[->, very thick] (tnfa) -- (bitmask);

  \end{tikzpicture}
\end{frame}

\begin{frame}{クリーネ閉包の遷移の模倣}
		正規表現の$0$回以上の繰り返し(クリーネ閉包)は,\\TNFA上で前の状態への遷移で表現
			\begin{itemize}
				\item Kaneta+[2010]は,バレルシフターによって,遷移を模倣
					\begin{description}[バレルシフター:]
						\item[バレルシフター:] 異なるビットを異なる数だけ\\ビットシフトを行うテクニック
					\end{description}
					\begin{itemize}
						\item 移動する回数を2進展開し,2冪毎にビットシフトを行う
					\end{itemize}
				\item 状態のビット列を反転し,伝搬を模倣することでも可能
					\begin{itemize}
						\item 反転する操作を効率的に行う必要がある
							\begin{itemize}
								\item {\oldfootnotesize CPUが持つSIMD命令で実現可能か考える}
							\end{itemize}
					\end{itemize}
			\end{itemize}
			\begin{tikzpicture}
				\node(tnfa) at (0,0) {\scalebox{.3}{\import{../figure/}{state_back}}};
				\node(inv) [right = of tnfa] {\scalebox{.3}{\import{../figure/}{state_inv}}};
				\draw[->, very thick] (tnfa) -- (inv);
			\end{tikzpicture}
\end{frame}
\end{document}
