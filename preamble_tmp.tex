\usepackage{tikz}
\usetikzlibrary{calc,decorations.pathreplacing,quotes,positioning,shapes,fit,arrows,backgrounds,tikzmark}
%\pgfdeclarelayer{background}
%\pgfsetlayers{background}
%%%% 和文用 %%%%%
\usepackage{bxdpx-beamer}
\usepackage{pxjahyper}
\usepackage{minijs}%和文用
\usepackage{latexsym}
\usepackage[deluxe,expert]{otf}
\renewcommand{\kanjifamilydefault}{\gtdefault}%和文用

%%%% スライドの見た目 %%%%%
\usetheme{Madrid}
\usefonttheme{professionalfonts}
\setbeamertemplate{frametitle}[default][center]
\setbeamertemplate{navigation symbols}{}
\setbeamercovered{transparent=0}%好みに応じてどうぞ)
\setbeamertemplate{footline}[page number]
\setbeamerfont{footline}{size=\footnotesize,series=\bfseries}
\setbeamercolor{footline}{fg=black,bg=black}
\usepackage{multicol}
\colorlet{mycolor}{orange!80!black}
\usepackage{caption}
\captionsetup[figure]{format=plain, labelformat=empty, labelsep=none}
\renewcommand{\figurename}{}
%%%%

%%%%% フォント基本設定 %%%%%
\usepackage[T1]{fontenc}%8bit フォント
\usepackage{textcomp}%欧文フォントの追加
\usepackage[utf8]{inputenc}%文字コードをUTF-8
\usepackage{txfonts}%数式・英文ローマン体を txfont にする
\usepackage{bm}%数式太字
%%%%%%%%%%

%%%% その他ほしかったもの%%%%
\usepackage{amsmath,amssymb}
\usepackage{amsthm}
\usepackage{algorithm, algpseudocode}
\newcommand{\func}[1]{\ensuremath\mathrm{#1}}
\newcommand{\Scatter}{\text{\scshape Scatter}}
\newcommand{\Gather}{\text{\textsc{Gather}}}
\newcommand{\Propagate}{\func{\text{\textsc{Propagate}}}}
%%%%%%%%

%%%% Standaloneかどうかでimportのパスを切り替える %%%%
\providecommand{\ImportStandalone}[3]{
	\IfStandalone{
		\import{#2}{#3}
		}{
		\import{#1#2}{#3}
	}
}
%%%%%%%%
